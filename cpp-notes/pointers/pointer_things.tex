\documentclass{report}

\input{~/latex/template/preamble.tex}
\input{~/latex/template/macros.tex}

\title{\Huge{Pointer Things}}
\author{\huge{Matt Warner}}
\date{\huge{}}
\pagestyle{fancy}
\fancyhf{}
\rhead{}
\lhead{\leftmark}
\cfoot{\thepage}
% \usepackage[default]{sourcecodepro} \usepackage[T1]{fontenc}

\pgfpagesdeclarelayout{boxed}
{
  \edef\pgfpageoptionborder{0pt}
}
{
  \pgfpagesphysicalpageoptions
  {%
    logical pages=1,%
  }
  \pgfpageslogicalpageoptions{1}
  {
    border code=\pgfsetlinewidth{1.5pt}\pgfstroke,%
    border shrink=\pgfpageoptionborder,%
    resized width=.95\pgfphysicalwidth,%
    resized height=.95\pgfphysicalheight,%
    center=\pgfpoint{.5\pgfphysicalwidth}{.5\pgfphysicalheight}%
  }%
}

\pgfpagesuselayout{boxed}

\begin{document}
    \maketitle
    \section{Using ptrdiff\_t for pointer subtraction}
    By definition, \texttt{ptrdiff\_t} can represent any possible result of subtracting two pointers that point to the same array.  Or, if you perfer, the distance between two elements of the same array in either direction.
    \nt{
        \texttt{ptrdiff\_t} is always a signed value.
        \bigbreak \noindent
        It's nearly always equivalent to \texttt{make\_signed\_t<size\_t>}
    }
    \noindent Although \texttt{ptrdiff\_t} is signed and \texttt{size\_t}, they're closely related. However, mixing them can easily lead to accidental signed-to-unsigned comparisions.
    \bigbreak \noindent
    When comparing signed and unsigned values of the same size, the compiler first converts the signed value to unsigned. Therefore, if the signed value is negative, this can produce suprising results.
    \section{Pointer types}
    When creating a pointer, we can do so in the following ways:
    \begin{minted}[linenos=false]{cpp}
    int* x = nullptr; // Pointer to an int
    const int* x = nullptr; // Pointer to a const int
    int* const x = nullptr // Const pointer to an int
    const int* const x = nullptr; // Const pointer to const int
    \end{minted}
    The following are the operations allowed/forbidden for these types
    \begin{minted}[linenos=false]{cpp}
    int x = 2;
    int y =3;
                                   
    // Pointer to an int 
    int* ptr = new int(4);
    delete ptr;
    ptr = &x; // OK
    *ptr = 3; // OK
              
    // Pointer to const int (address pointed to can be changed (value stored at that address cannot) )
    const int* ptr2 = &x;
    ptr2 = &y; // OK
     *ptr2 = 3; // Error!

    // const pointer to int (address cannot be changed but the value stored at address can)
    int* const ptr3 = &x;
    *ptr3 = 3; // OK
     ptr3 = &y; // Error!

    // const pointer to a const int (address cannot be changed nor can the value stored at that address)
    const int* const ptr4 = &x;
    *ptr3 = 3; // Error!
     ptr3 = &y // Error;

    \end{minted}
    \section{Function pointers}
    Functions have memory addresses, so its only natural that we can create pointers to functions. The syntax looks like this:
    \begin{minted}[linenos=false]{text}
   return_type (*Funcptr) (parameter type, .....);
    \end{minted}
    Now we can give it an address to a function. Note: you cannot declare and initalize function pointers at the same time. You must first declare and then initalize. Consider the following code:
    \begin{minted}[linenos=false]{cpp}
    void foo() { std::cout << "Hello from foo. "; }

    int main() {

        // Declaring a function pointer
        void (*funcptr)();
        
        // funcptr is pointing to foo
        funcptr = foo;

        // calling the function.
        funcptr();

        return 0;
    }
    \end{minted}
    \subsection{Passing a function pointer as a parameter}
    \begin{minted}[linenos=false]{cpp}
    void print(void (*funcptr)()) {
        std::cout << funcptr();
    }
    int main () {
        void (*funcptr)();
        funcptr = foo;
        print(funcptr);
    }
    \end{minted}
\end{document}
