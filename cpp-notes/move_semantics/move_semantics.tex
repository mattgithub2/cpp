\documentclass{report}

\input{~/latex/template/preamble.tex}
\input{~/latex/template/macros.tex}

\title{\Huge{Move semantics}}
\author{\huge{Matt Warner}}
\date{\huge{}}
\pagestyle{fancy}
\fancyhf{}
\rhead{}
\lhead{\leftmark}
\cfoot{\thepage}
% \usepackage[default]{sourcecodepro}
% \usepackage[T1]{fontenc}

\usepackage{tikz}
\usepackage{pgfplots}
\pgfplotsset{compat=1.18}

\pgfpagesdeclarelayout{boxed}
{
  \edef\pgfpageoptionborder{0pt}
}
{
  \pgfpagesphysicalpageoptions
  {%
    logical pages=1,%
  }
  \pgfpageslogicalpageoptions{1}
  {
    border code=\pgfsetlinewidth{1.5pt}\pgfstroke,%
    border shrink=\pgfpageoptionborder,%
    resized width=.95\pgfphysicalwidth,%
    resized height=.95\pgfphysicalheight,%
    center=\pgfpoint{.5\pgfphysicalwidth}{.5\pgfphysicalheight}%
  }%
}

\pgfpagesuselayout{boxed}

\begin{document}
    \maketitle
\section{lvalues and rvalues}
The term ``lvalue'' stands for ``locator value''. In early C, an lvalue was considered to be something that could appear on the left-hand side of an assignment operator (`='). It was named as such because it represents a value that can be located in memory.
\bigbreak \noindent
The term ``rvalue'' stands for ``right-hand side value''. In early C, an rvalue was something that could appear only on the right-hand side of an assignment operator (`='). Rvalues were typically temporary values that did not have a persistant memory location.
\bigbreak \noindent
Heres a simplifed Definition for lvalues and rvalues:

\begin{itemize}
  \item \textbf{lvalue} - An object that occupies some identifiable location in memory
  \item \textbf{rvalue} - Temporary values that are used in expressions but do not persist beyond the expression in which they appear. 
\end{itemize}
\subsubsection*{lvalue examples}
\begin{minted}[linenos=true]{cpp}
  int i;    // i is a lvalue 
  int* p = &i; // i's address is identifiable
  i = 2; // Memory content is modified

  class dog;  // lvalue of user defined type (class)
  dog dl;      
  
\end{minted}
\nt{
  Most variables in C++ code are lvalues
}
\subsubsection*{rvalue examples}
\begin{minted}[linenos=true]{cpp}
  int x = 2; // 2 is an rvalue
  int x = i+2;  // (i+2) is an rvalue
  int* p = &(i+2); // ERROR
  i+2 = 4; // ERROR
  2 = i; // ERROR

  dog d1;
  d1 = dog(); // dog() is rvalue of user defined type (class)

  int sum(int x, int y) { return x+y; }
  int i = sum(3,4); // sum(3,4) is rvalue
\end{minted}
\subsubsection*{l-value references}
the term ``l-value reference'' just refers to references in C++. Here are some examples of those:
\begin{minted}[linenos=true]{cpp}
int i;
int &r = i; // r is a reference to i;

int &r = 5; // Error

// Exception: Constant lvalue reference can be assigned a rvalue
const int &r = 5;


int square(int& x) { return x * x; }

square(i); // OK
square(40); // Error!

// Workaround:
int square(const int& x) { return x*x; } // square(40) and square(i) both work with this

\end{minted}
In summary, reference variables can only refer to lvalues, except for const reference variables. They can refer to rvalues.

\subsection{r-value references}
An r-value referene in C++ is a reference type that was introduced to support move semantics, which allows efficient transfer of resources.
\nt{
    The transfer of ownership facilitated by move semantics primarily applies to resources managed on the heap.
}
\bigbreak \noindent
To understand the basic concept of r-value refernces, lets take a look at this example:
\begin{minted}[linenos=true]{cpp}
  void print_int(int &i) { std::cout << "l-value reference: " << i << std::endl} // takes a l-value reference to i as parameter
  void print_int(int &&i) { std::cout << "r-value reference: " << i << std::endl} // takes a r-value reference  to i as parameter
\end{minted}
The following code demonstrates how to call both functions.
\begin{minted}[linenos=false]{cpp}
  int a = 5;
  print_int(a) // calls l-reference function (called with an l-value)
  print_int(10) // calls r-reference function (called with an r-value)
\end{minted}
As you can see, in its simplest form, functions with r-value reference parameters expect an r-value, and l-value references expect an l-value. To understand why r-value references are used to allow efficient tranfer of resources, we must take a look at \textbf{std::move}.
\bigbreak \noindent
std::move is a utility function provided by C++ to facilitate move semantics. Its purpose is to cast a given argument into an r-value reference, which allows the programmer to indicate that they are potentially willing to ``move'' resources from the object being passed in. Take a look at this example:
\begin{minted}[linenos=true]{cpp}
  std::vector<int> vec1{1,2,3,4};
  std::vector<int> vec2{5,6,7,8};

  // Assigns copys vec2 into vec1
  vec1 = vec2;
\end{minted}
In this example, we want vec1 to have the same contents of vec2. Without using std::move,  we need copy the contents of vec2 and give them to vec1. However, using std::move, we can tranfer the resources to vec1 instead.
\begin{minted}[linenos=true]{cpp}
  vec1 = std::move(vec2);
\end{minted}
Now, instead of of coping over the contents of the vector, we are tranfering ownership of vec2's resources from vec2 to vec1, which is less costly.
\nt{
    After std::move, vec2 is in a valid but unspecified state and should not be used.
}
\end{document}
