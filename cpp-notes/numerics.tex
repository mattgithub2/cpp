\documentclass{report}

\input{~/latex/template/preamble.tex}
\input{~/latex/template/macros.tex}

\title{\Huge{Numerical algorithms}}
\author{\huge{Matt Warner}}
\date{\huge{}}
\pagestyle{fancy}
\fancyhf{}
\rhead{}
\lhead{\leftmark}
\cfoot{\thepage}
% \usepackage[default]{sourcecodepro} \usepackage[T1]{fontenc}

\pgfpagesdeclarelayout{boxed}
{
  \edef\pgfpageoptionborder{0pt}
}
{
  \pgfpagesphysicalpageoptions
  {%
    logical pages=1,%
  }
  \pgfpageslogicalpageoptions{1}
  {
    border code=\pgfsetlinewidth{1.5pt}\pgfstroke,%
    border shrink=\pgfpageoptionborder,%
    resized width=.95\pgfphysicalwidth,%
    resized height=.95\pgfphysicalheight,%
    center=\pgfpoint{.5\pgfphysicalwidth}{.5\pgfphysicalheight}%
  }%
}

\pgfpagesuselayout{boxed}

\begin{document}
    \maketitle
\noindent Contained the the header file: \texttt{numeric}, are a couple of benefical functions. Here are the four major ones:
\section{std::accumlate}
Computes the sum (by default) of the given value \texttt{init} and the elements in the range [first, last), and returns the computed sum. \texttt{std::accumulate} has one definition that simply takes iterators \texttt{first} and \texttt{last}, and \texttt{init}, which is used to initialize the accumulator \texttt{acc}. The other definition of \texttt{std::accumulate} takes the same parameters as the previous definition, but adds an additional fourth parameter, \texttt{op}. Which is a \textit{binary function object} that is applied. This binary function object can be used to perform other operators on the container like multiply, etc.
\bigbreak \noindent
The below snippet are the two declarations.
\begin{minted}[linenos=false]{cpp}
template< class InputIt, class T >
T accumulate( InputIt first, InputIt last, T init );   (1)

template< class InputIt, class T, class BinaryOp >
T accumulate( InputIt first, InputIt last, T init, BinaryOp op );  (2)
\end{minted}
Note that if any of the following conditions is satisfied, the behavior is undefined.
\begin{itemize}
    \item \texttt{T} is not \textit{CopyConstructable}
    \item \texttt{T} is not \textit{CopyAssignable}
    \item \texttt{op} modifies any elements of [first, last)
    \item \texttt{op} invalidates any iterator or subrange in [first, last).
\end{itemize}
        \bigbreak \noindent
        Also note that the signature of \texttt{op} should be equivalent to the following (doesn't strictly need to be const \&).
        \begin{minted}[linenos=false]{cpp}
        Ret fun(const Type1&, const Type2 &b);
        \end{minted}
        \noindent Further note that the type \texttt{Type1} must be such that an object of type \texttt{T} can be implicitly converted to \texttt{Type1}. The type \texttt{Type2} must be such that an object of type \texttt{InputIt} can be dereferenced and then implicitly converted to \texttt{Type2}. Lastly, the type \texttt{Ret} must be such that an object of type \texttt{T} can be assigned a value of type \texttt{Ret}.
        \bigbreak \noindent
    The behavior of this function template is equivalent to:
    \begin{minted}[linenos=false]{cpp}
    template <typename InputIterator, typename T> 
    T accumulate (InputIterator first, InputIterator last, T init) {
        while (first != last) {
            init = init + *first;
            ++first;
        }
        return init;
    }
    \end{minted}
    Or if you provide a function object:
    \begin{minted}[linenos=false]{cpp}
     
    template <typename InputIterator, typename T> 
    T accumulate (InputIterator first, InputIterator last, T init) {
        while (first != last) {
            init = binary_op(init, *first);
            ++first;
        }
        return init;
    }
    \end{minted}
Also note that the header file <functional> provides a couple of function objects that are handy for something like \texttt{std::accumulate}. These include:
\begin{minted}[linenos=false]{cpp}
std::multiplies<T>();
std::minus<T>();
std::plus<T>();
std::divides<T>();
std::modulus<T>();
\end{minted}
Heres an example program that uses \texttt{std::accumulate}
\begin{minted}[linenos=false]{cpp}
#include <functional>
#include <iostream>
#include <numeric>
#include <string>
#include <vector>

int main() {
    std::vector<int> v{1,2,3,4,5,6,7,8,9,10};
    int sum = std::accumulate(v.begin(), v.end(), 0);
    int product = std::accumulate(v.begin(), v.end(), 1, std::multiplies<int>());

    auto dash_fold = [](std::string a, int b) {
        return std::move(a) + " - " + std::to_string(b);
    }
    std::string s = std::accumulate(std::next(v.begin()), v.end(), )
}
\end{minted}
    \section{reduce (c++17)}
    The \texttt{reduce()} method in C++ is used for applying an algorithm to a range of elements in an array. By default, it returns the sum of values of elements in the applied range. It behaves similarly to \texttt{std::accumulate in STL}.
    \bigbreak \noindent
    There are a variety of overloads for the reduce function. We have:
    \begin{minted}[linenos=false]{cpp}
template<class Inputit>
typename std::iterator_traits<InputIt>::value_type
reduce(InputIt first, InputIt last);
     
template<class ExecutionPolicy, class ForwardIt>
typename std::iterator_traits<ForwardIt>::value_type
reduce(ExecutionPolicy&& policy, ForwardIt first, FowardIt last);
    
template<class InputIt, class T>
T reduce(InputIt first, InputIt last, T init);

template<class ExecutionPolicy, class ForwardIt, class T>
T reduce(Execution Policy, ForwardIt first, Forward It last, T init);

template< class InputIt, class T, class BinaryOp >
T reduce( InputIt first, InputIt last, T init, BinaryOp op );
    
template<class ExecutionPolicy, class T, class BinaryOp>
typename std::iterator_traits<ForwardIt>::value_type
T reduce(ExecutionPolicy&& policy, ForwardIt first, ForwardIt last, T init, BinaryOp op);
    \end{minted}
    \subsection*{Parameters}
    \textit{\textbf{first, last}}  \ - \ the range of elements to apply the algorithm to \vspace{1.5mm} \\ 
    \textit{\textbf{init}} \ - \ the initial value of the generalized sum \vspace{1.5mm} \\
    \textit{\textbf{policy}} \ - \ the execution policy to use. \vspace{1.5mm} \\
    \textit{\textbf{op}} \ - \ binary \textit{function object} that will be applied in unspecified order to the result of dereferencing the input iterators, the result of other \texttt{op} and \texttt{init}
\subsection*{Return value}
Returns the value of the result of the reduction operation.
\subsubsection*{Execution Policies}
Execution policies are a C++17 feature which allows programmers to ask for algorithms to be parallelised. These are three execution policies in C++17:
\begin{itemize}
    \item \texttt{std::execution::seq} \ - \ do not parallelise
    \item \texttt{std::execution::par} \ - \ parallelise
    \item \texttt{std::exection::par\_unseq} \ - \ parallelise and vectorise (requires that the operation can be interleaved, so no acquiring mutexes and such)
\end{itemize} 
\subsection*{Example:} 
\begin{cppcode}
  
int main() {
    std::vector<int> v = {1,2,3,4,5};

    // reduce returns sum of the elements in the given range
    int sum = reduce(v.begin(), v.end(), 0); 
    std::cout << "Default execution of the reduce function: " <<  sum << std::endl;

    // here it returns the sum without needing to pass the initial value
    sum = reduce(v.begin(), v.end());
    cout << "Execution with default initial value: " << sum << std::endl;
}
\end{cppcode}

\end{document}
