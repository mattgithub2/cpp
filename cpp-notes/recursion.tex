\documentclass{report}

\input{~/latex/template/preamble.tex}
\input{~/latex/template/macros.tex}

\title{\Huge{Recursion Notes}}
\author{\huge{Matt Warner}}
\date{\huge{}}
\pagestyle{fancy}
\fancyhf{}
\rhead{}
\lhead{\leftmark}
\cfoot{\thepage}
% \usepackage[default]{sourcecodepro}
% \usepackage[T1]{fontenc}

\usepackage{tikz}
\usepackage{pgfplots}
\pgfplotsset{compat=1.18}

\pgfpagesdeclarelayout{boxed}
{
  \edef\pgfpageoptionborder{0pt}
}
{
  \pgfpagesphysicalpageoptions
  {%
    logical pages=1,%
  }
  \pgfpageslogicalpageoptions{1}
  {
    border code=\pgfsetlinewidth{1.5pt}\pgfstroke,%
    border shrink=\pgfpageoptionborder,%
    resized width=.95\pgfphysicalwidth,%
    resized height=.95\pgfphysicalheight,%
    center=\pgfpoint{.5\pgfphysicalwidth}{.5\pgfphysicalheight}%
  }%
}

\pgfpagesuselayout{boxed}

\begin{document}
  \maketitle

\section{Overview}

Recursion is a general programming technique used to solve problems with a ``divide and conquer'' strategy
\bigbreak \noindent
\nt{
  most computer programming langauges support recursion by allowing a function or member function to call itself within the program text.

}
\bigbreak \noindent
\textbf{Example of recursion}
\begin{cppcode}
int factorial(int n)
{

  if (n == 1)
  {
  return 1 ;

  }
  else

  {
  return n * factorial(n-1);
  }

}
\end{cppcode}
\bigbreak \noindent
A recursive function call will always by conditional. There must tbe at least one base case for which the function produces a result trivially without a recursive call. For example:
\begin{cppcode}
int factorial(int n){

  if (n == 1 ){   // Base case - no recursion
    return 1;
  }
  else
  {
    return n * factorial(n-1);
  }
}
\end{cppcode}
\bigbreak \noindent
A function with no base cases leads to ``infinite recursion'' (similar to an infinite loop)
\bigbreak \noindent
In addition to the base cases, a recursive function will have one or more recursive cases. The job of a recursive case can be seen as breaking down complex inputs into simpler ones.
\bigbreak \noindent
In a properly designed recursive function, with each recursive call, the input problem must be simplified in such a way that eventually the base case must be reached. For example:
\begin{cppcode}
  int factorial(int n) {
    if (n == 1) {
      return 1;
    }
    else
      return n * factorial(n - 1); // n - 1 gets us closer to 1
  }
\end{cppcode}

\newpage
\noindent Recursion requires automatic storage - each new call to the function creates its own copies of the local variables and function parameters on the program stack.
\bigbreak \noindent
Recursion is never a required programming technique. Recursion can always be replaced by either
\begin{enumerate}
  \item A loop 
  \item A loop and a stack (that takes the place of the program call stack used by recursion)
\end{enumerate}
In the first case, a non-recursive algorithm will usually be superior, if only in terms of memory usage. But in the second case, we might choose a recursive algorith if it is shorter or eaiser to code (which is often true).
\section{Recursive Trees}
A \textit{recursion tree} is useful for visualizing what happens when a recurrence is iterated. It diagrams the tree of recursive calls and the amount of work done at each call.
\bigbreak \noindent
For instance, consider the following recursive function:
\begin{cppcode}

  int fib(int n) {

    if (n < 2)
      return n;
    return fib(n - 1) + fib(n - 2);
  }

\end{cppcode}
\bigbreak \noindent 
Here is an example recursive tree for \textbf{fib(4)}, note the repeated computations:

\begin{figure}[H]
\centering
\includegraphics[width=0.6\textwidth]{ ~/figures/fib.jpg}
\end{figure}

\section{Tail Recursion}
\textit{Tail recursion} is defined as a recursive function in which the recursive call is the last statement that is executed by the function. So basically, nothing is left to execute after the recursive call.
\bigbreak \noindent
\begin{cppcode}

  void print(int n) {

    if (n < 0) {
      return;
    }
    // The last executed statement is recursive call
    print(n - 1);
}
\end{cppcode}
\subsection{Need for Tail Recursion}
The tail recursive functions are considered better than non-tail recursive functions as tail-recursion can be optimized by the compiler.
\bigbreak \noindent
Compilers usually execute recursive procedures by using a \textit{stack}. This stack consits of all the pertinent information, including the parameter values for each recursive call.
\bigbreak \noindent
When a procedure is called, its information is \textit{pushed} onto a stack, and when the function terminates the information is \textit{popped} out of the stack. Thus for the non-tail recursive functions, the \textit{stack depth} (maximum amount of stack space used at any time during compilation) is more.
\subsubsection*{Can a non-tail-recursive function be written as tail-recursive to optimize it?}
Consider the following function to calculate the factorial of n.
\begin{cppcode}
  unsigned int fac(usigned int n) {
    if (n <= 0)
      return 1;
    return n * fac(n - 1);
  }
\end{cppcode}
\bigbreak \noindent Although this looks like tail-recursion, if we take a closer look we can see that the value returned by \textbf{fac(n - 1)} is used in \textbf{fac(n)}. So the call to \textbf{fac(n-1)} is not the last thing done by \code{fac(n)}.
The above function can be written as a tail-recursive function. The idea is to use one more argument and accumulate the factorial value in the second argument. When n reaches 0, return the accumulated value.
\bigbreak \noindent
\begin{cppcode}
  unsigned factTR(unsigned int n, unsigned int a) {
    if (n <= 1)
      return a;
    return factTR(n - 1, n * a);
  }

  // Wrapper over factTR
  unsigned int fact(unsigned int n) { return facttr(n,1); }
\end{cppcode}
\bigbreak \noindent
\end{document}
