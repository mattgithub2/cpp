\documentclass{report}

\input{~/latex/template/preamble.tex}
\input{~/latex/template/macros.tex}

\title{\Huge{Lambda in C++}}
\author{\huge{Matt Warner}}
\date{\huge{}}
\pagestyle{fancy}
\fancyhf{}
\rhead{}
\lhead{\leftmark}
\cfoot{\thepage}
% \usepackage[default]{sourcecodepro}
% \usepackage[T1]{fontenc}

\usepackage{tikz}
\usepackage{pgfplots}
\pgfplotsset{compat=1.18}

\pgfpagesdeclarelayout{boxed}
{
  \edef\pgfpageoptionborder{0pt}
}
{
  \pgfpagesphysicalpageoptions
  {%
    logical pages=1,%
  }
  \pgfpageslogicalpageoptions{1}
  {
    border code=\pgfsetlinewidth{1.5pt}\pgfstroke,%
    border shrink=\pgfpageoptionborder,%
    resized width=.95\pgfphysicalwidth,%
    resized height=.95\pgfphysicalheight,%
    center=\pgfpoint{.5\pgfphysicalwidth}{.5\pgfphysicalheight}%
  }%
}

\pgfpagesuselayout{boxed}

\begin{document}
	\maketitle
	\section{Creating a Lambda expression in C++} 
  C++ 11 introduced lambda expressions to allow inline functions which can be used for short snippets of code that are not going to be reused and therefore do not require a name. In their simplest form a lambda expression can be defined as follows:
  \begin{minted}[linenos=false]{text}
    [ caputure clause ]  ( parameters ) -> return-type
    {
      definition of method 
    }
  \end{minted}
  \bigbreak \noindent
  A basic lambda expression can look something like this:
  \begin{cppcode}
    auto greet = [] () {
      // Lambda function body
    };
  \end{cppcode}
  \bigbreak \noindent
  Generally, the return-type in lambda expressions is evaluated by the compiler itself and we don't need to specify it explicitly. Also the $\rightarrow$ return-type part can be ignored. However, in some complex cases e.g. conditional statements, the compiler can't determine the return type and explicit specifications is required.
\bigbreak \noindent
  Here are the parts of the lambda shown above:
  \begin{itemize}
    \item $[ \ ]$ is called the \textbf{\textit{Lambda introducer}} which denotes the start of the lambda expression.
    \item () is called the \textbf{\textit{parameter list}} which is similar to the () operator of a normal function.
  \end{itemize}
 \bigbreak \noindent \bigbreak \noindent
  The above code is equivalent to:
  \begin{cppcode}
    void greet() {
      // function body
    }
  \end{cppcode}
  \bigbreak \noindent Now, just like a normal function, we can simply invoke the lambda expression using:
  \textit{greet();}
  \nt{
    We have used the \textit{\textbf{auto}} keyword to automatically deduce the return type for the lambda expression.
  }
  \section{Example:}
  \begin{cppcode}
   #include <iostream>
   
   int main() {

     // Create a lambda function that prints "Hello World!"
     auto greet = []() {
       std::cout << "Hello World!";
     };

     // call lambda function
     greet();

     return 0;
   }

  \end{cppcode}
  \section{Lambda with parameters}
  Just like a regular function, lambda expressions can also take parameters. For example, 
  \begin{cppcode}
   #include <iostream> 

   int main() {

     auto add = [](int a, int b) {
        std::cout << "Sum = " << a + b;
     };

     add(100, 78);
   }
  \end{cppcode}
  \bigbreak \noindent
  In the above example, we have created a lambda function to which takes two integer parameters and displays their sum.
  \bigbreak \noindent
  This is equivelent to:
  \begin{cppcode}
    void add(int a, int b) {
      std::cout << "Sum = " << a + b;
    }
  \end{cppcode}
  \section{Lambda with return type}
  Like with normal functions, c++ lambda expressions can also have a return type.
  \bigbreak \noindent
  The compiler can implicitly deduce the return type of the lambda expression based on the return statements.
  \begin{cppcode}
    auto add = [](int a, int b) {
      // always returns 'int'
      return a + b;
    };
  \end{cppcode}
  \noindent In the above case, we have not explicitly defined the return type for the lambda function. This is because there is a single return statement which always returns an integer value.
\bigbreak \noindent
But for multiple return statements of different types, we have to explicitly define the type. For example.
\begin{cppcode}
  auto operation = [](int a, int b, string op) -> double {
    if (op == "sum") {
      // return an int value
      return a + b;
    } else {
      // return a double
      return (a+b) / 2.0;
    }
  }
\end{cppcode}
\noindent Notice the code \textbf{ $\rightarrow$ double}. This explicitly defines the return type as a \textbf{double}, since there are multiple return statements which return different types based on the value of \textbf{op}.
\section{Lambda Function Capture Clause}
By default, lambda functions cannot access variables of the enclosing function. In order to access those variables, we use the capture clause. \vspace{1.5mm}

\noindent We can capture the variables in two ways:
\subsection*{Capture by Value}
This is similar to \textbf{calling a function by value}. Here, the actual value is copied when the lambda is created.
\nt{
  Here, we can only read the variable inside the lambda body but cannot modify it.
}
\bigbreak \bigbreak \noindent A basic lambda expression with caputre by value looks as follows:
\begin{cppcode}
 int num_main = 100; 

 // get access to num_main from the enclosing function
 auto my_lambda = [num_main] () {
   cout << num_main;
 };
\end{cppcode}
\bigbreak \noindent
\bigbreak \noindent
Here, \textbf{$[$num\_main$]$} allows the lambda to access the \textbf{num\_main} variable.
\bigbreak \noindent
If we remove \textbf{num\_main} from the capture clause, we will get an error since \textbf{num\_main} cannot be accessed from the lambda body.
\subsection*{Capture by reference}
This is similar to \textbf{calling a function by reference} i.e. the lambda has access to the variable address.
\bigbreak \noindent
A basic lambda expression with capture by reference looks as follows.
\begin{cppcode}
 int num_main = 100; 

 auto my_lambda = [&num_main] () {
    num_main = 900;
 };
\end{cppcode}
\section{Various uses for lambda expressions}
Here's one example of when lambda expressions are useful:
\begin{cppcode}
  #include <algorithm>
  #include <vector>

  // Function to print vector
  void printVector(vector<int> v) {

    // Lambda expression to print vector
    for_each(v.begin(), v.end(), [](int i)) 
    {
      std::cout << i << " ";
    });
    std::cout << std::endl;
  }
\end{cppcode}
\end{document}
